\documentclass{article}

% \prooftree and related commands
\usepackage{ebproof}
% most symbols
\usepackage{amsmath}
\usepackage{amssymb}
\usepackage{amsthm}
% \RHD
\usepackage[nointegrals]{wasysym}
\usepackage{xcolor}

\newcommand{\ignore}[1]{}
\newcommand{\todo}[1]{\colorbox{red}{#1}}
\newcommand{\consider}[1]{\colorbox{red}{!!!} #1 \colorbox{red}{!!!}}

\newtheorem{theorem}{Theorem}
\newtheorem{lemma}{Lemma}

\newcommand{\define}{::=}
\newcommand{\G}{\Gamma}

\renewcommand{\implies}{\Rightarrow}
\newcommand{\variant}[1]{[\!|#1|\!]}
\newcommand{\case}[2]{\tt{case}\,#1\,#2}
\newcommand{\app}[2]{#1\,#2}
\newcommand{\lam}[2]{\lambda #1 . #2}

\newcommand{\hastp}[3]{#1 \vdash #2 \,:\, #3}
\newcommand{\haslbl}[3]{#1\;|\;#2\,:\, #3}
\newcommand{\caseVarRcd}[3]{#1 \;\sqsubseteq\; #2 \,:\, #3}
\newcommand{\projFun}{\,\langle\!\!\#\,}
\newcommand{\projArg}{\,\#\!\!\rangle\,}
\newcommand{\projFunRule}[3]{#1 \;\trianglelefteq\; #2 \,@\, #3}
\newcommand{\projArgRule}[3]{#1 \;\trianglerighteq\; #2 \,@\, #3}

\newcommand{\haslbltm}[3]{#1\,|\,#2\,\hookrightarrow\, #3}
\newcommand{\eval}[2]{#1 \hookrightarrow #2}
\newcommand{\step}[2]{#1 \leadsto #2}
\newcommand{\steps}[2]{#1 \leadsto^* #2}

\renewcommand{\impliedby}{\Leftarrow}
\newcommand{\B}{\mathcal{B}}
\newcommand{\C}{\mathcal{C}}
\newcommand{\bottom}{\perp}
\newcommand{\marker}[1]{\RHD_{#1}}
\newcommand{\subtype}{<:}
\newcommand{\multisubtype}{<\!\!<:}
\newcommand{\multisuptype}{:>\!\!>}
\newcommand{\synthesizes}{\Rightarrow \!\!\! \Rightarrow}
%\newcommand{\app}{\bullet}
\newcommand{\prj}{\,\pmb{\#}\,}
\newcommand{\instL}{\;\substack{<\\:=}\;}
\newcommand{\instR}{\;\substack{<\\=:}\;}
\newcommand{\ev}{\hat}
\newcommand{\spc}{\qquad}
\newcommand{\apply}[1]{\left[#1\right]}

\newcommand{\rnil}{\{\}}
\newcommand{\rcons}[2]{\{#1 \,|\, #2\}}

\newcommand{\synth}[4]{#1 \vdash #2 \Rightarrow #3 \dashv #4}
\renewcommand{\check}[4]{#1 \vdash #2 \Leftarrow #3 \dashv #4}

\newcommand{\presynth}[6]{#1 \vdash #2 #3 #4 \synthesizes #5 \dashv #6}

\newcommand{\subtypes}[4]{#1 \vdash #2 \subtype #3 \dashv #4}
\newcommand{\multisubtypes}[4]{#1 \vdash #2 \multisubtype #3 \dashv #4}
\newcommand{\subtypesmulti}[4]{#1 \vdash #3 \multisuptype #2 \dashv #4}

\newcommand{\refinesL}[4]{#1 \vdash #2 \instL #3 \dashv #4}
\newcommand{\refinesR}[4]{#1 \vdash #2 \instR #3 \dashv #4}

\newcommand{\Glb}[4]{#1 \vdash \bigvee #2 \synthesizes #3 \ignore{\dashv #4}}
\newcommand{\Lub}[4]{#1 \vdash \bigwedge #2 \synthesizes #3 \ignore{\dashv #4}}
%\newcommand{\glbsynth}[5]{#1 \vdash #2 \glb #3 \synthesizes #4 \ignore{\dashv #5}}
%\newcommand{\lubsynth}[5]{#1 \vdash #2 \lub #3 \synthesizes #4 \ignore{\dashv #5}}
\newcommand{\glbsynth}[5]{#2 \glb #3 \synthesizes #4 \ignore{\dashv #5}}
\newcommand{\lubsynth}[5]{#2 \lub #3 \synthesizes #4 \ignore{\dashv #5}}
\newcommand{\glb}{\vee}
\newcommand{\lub}{\wedge}
\newcommand{\nilC}{\varnothing}
\newcommand{\set}[1]{#1}

\newcommand{\lookup}[5]{#1 \vdash #2 \# #3 \longrightarrow #4 \dashv #5}

\newcommand{\deduct}[3][]
{
  \begin{prooftree}
    \hypo{#2}
    \infer1[\text{#1}]{#3}
  \end{prooftree}
}

\begin{document}

\section{Grammar}

\subsection{Types}
\[\begin{array}{rcl}
\tau & \define & \alpha \mid \forall \alpha. \tau \mid \tau_0 \to \tau_1 \mid \{\rho\} \mid \variant{\rho}
\\
\rho & \define & \cdot \mid \_ : \tau \mid l : \tau, \rho
\\
l & \define & \text{row field labels}
\\
\alpha & \define & \text{bound type variables}
\end{array}
\]

$FV(\tau)$ denotes the free type variables of $\tau$ defined in the
usual way where the only type variable binder is $\forall\alpha.\tau$.

A row is {\it closed} if it ends in $\cdot$. We abuse $,$ to stand
for both adding a label to a row (e.g. $l:\tau,\rho$) and {\it appending}
two rows together when the first row is closed; e.g.
\[
(l_0:\tau_0,\l_1:\tau_1,\cdot),(l_2:\tau_2,l_3:\tau_3,\_:\tau)
\quad
\equiv
\quad
l_0:\tau_0,\l_1:\tau_1,l_2:\tau_2,l_3:\tau_3,\_:\tau
\]
Furthermore, writing $\rho_0,\rho_1$ implies $\rho_0$ is closed,
so $\rho,\_:\tau$ implies $\rho$ is closed and $\rho,\cdot
\equiv \rho$.

We do not forbid multiple occurrences of the same label in a row;
outer (to the left) occurrences shadow inner occurences as specified
in~\ref{subsec:type-lookup}.

\subsection{Type Substitution}
\[
\begin{array}{rcl}
  \alpha[\alpha := \tau] & = & \tau \\
  \alpha'[\alpha := \tau] & = & \alpha' \quad(\alpha \neq \alpha') \\
  (\forall \alpha. \tau')[\alpha := \tau] & = & \forall\alpha.\tau'\\
  (\forall \alpha'. \tau')[\alpha := \tau] & = & \forall \alpha'' . \tau'[\alpha' := \alpha''][\alpha := \tau] \quad (\alpha \neq \alpha' \,\wedge\, \alpha\neq\alpha'' \,\wedge\, \alpha'' \not\in FV(\tau))\\
  (\tau_0 \to \tau_1)[\alpha := \tau] & = & \tau_0[\alpha:=\tau]\to\tau_1[\alpha:=\tau]\\
  \{\rho\}[\alpha := \tau] & = & \{\rho[\alpha:=\tau]\}\\
  \variant{\rho}[\alpha := \tau] & = & \variant{\rho[\alpha:=\tau]} \\
  \\
  (\cdot)[\alpha := \tau] & = & \cdot \\
  (\_ : \tau')[\alpha := \tau] & = & \_ : \tau'[\alpha := \tau] \\
  (l:\tau', \rho)[\alpha := \tau] & = & l:\tau'[\alpha:=\tau],\rho[\alpha:=\tau]
\end{array}
\]

\subsection{Terms}

\[\begin{array}{rcl}
e & \define & x \mid \lam{x}{e} \mid \app{e_0}{e_1} \mid \{r\} \mid e\#l \mid \variant{l:e} \mid \case{e_0}{e_1}
\\
r & \define & \cdot \mid \_ : e \mid l : e, r
\\
x & \define & \text{bound term variables}
\end{array}
\]

$FV(e)$ denotes the free term variables of $e$ defined in the usual way where the only variable binder is $\lam{x}{e}$.

\subsection{Term Substitution}
\[
\begin{array}{rcl}
  x[x:=e] & = & e \\
  x' [x:=e] & = & x' \quad (x\neq x') \\
  (\lam{x}{e'})[x:=e] & = & \lam{x}{e'} \\
  (\lam{x'}{e'})[x:=e] & = & \lam{x''}e'[x':=x''][x:=e] \quad (x\neq x' \,\wedge\, x\neq x'' \,\wedge\, x''\not\in FV(e)) \\
  (\app{e_0}{e_1})[x:=e] & = & \app{e_0[x:=e]}{e_1[x:=e]} \\
  \{r\}[x:=e] & = & \{r[x:=e]\} \\
  (e'\#l)[x:=e] & = & (e'[x:=e])\#l \\
  \variant{l:e'}[x:=e] & = & \variant{l:e'[x:=e]} \\
  (\case{e_0}{e_1})[x:=e] & = & \case{e_0[x:=e]}{e_1[x:=e]} \\
  \\
  (\cdot)[x:=e] & = & \cdot \\
  (\_ : e')[x:=e] & = & \_ : e'[x:=e] \\
  (l : e', r)[x:=e] & = & l:e'[x:=e], r[x:=e]
\end{array}
\]

\subsection{Contexts}
\[\begin{array}{rcl}
\G & \define & \cdot \mid x : \tau, \G
\end{array}
\]
We assume that well formed contexts have a single occurrence of any variable.

\section{Typing}

\subsection{Type Label Lookup \;($\haslbl{\rho}{l}{\tau}$)}
\label{subsec:type-lookup}
\[
\deduct
    {}
    {\haslbl{\_ : \tau}{l}{\tau}}
    {}
\qquad
\deduct
    {}
    {\haslbl{l:\tau,\rho}{l}{\tau}}
    {}
\qquad
\deduct
    {\haslbl{\rho}{l}{\tau}}
    {\haslbl{l':\tau',\rho}{l}{\tau}}
    {(l \neq l')}
\qquad
\deduct
    {\haslbl{\rho}{l}{\forall\alpha.\tau}}
    {\haslbl{\rho}{l}{\tau[\alpha := \tau']}}
    {}
\]

Note that we allow instantiation of polymorphic fields during type
label lookup to facilitate the formulation of the constraint, in~\ref{subsec:case-constraint}, on
the rows involved in a $\case{e_0}{e_1}$ expression.

\subsection{Case Constraint \;($\caseVarRcd{\rho_0}{\rho_1}{\tau}$)}
\label{subsec:case-constraint}
\[
\deduct
    {\haslbl{\rho_1}{l_0}{\tau_0\to\tau} \quad \caseVarRcd{\rho_0}{\rho_1}{\tau}}
    {\caseVarRcd{l_0:\tau_0,\rho_0}{\rho_1}{\tau}}
    {}
\qquad
\deduct
    {}
    {\caseVarRcd{\cdot}{\rho}{\tau}}
    {}
\qquad
\deduct
    {\haslbl{\rho}{l}{\tau_0\to\tau}}
    {\caseVarRcd{\_:\tau_0}{\rho}{\tau}}
    {(l\not\in\rho)}
%\deduct
%    {}
%    {\caseVarRcd{\_:\tau_0}{\_ : \tau}{\tau}}
%    {}
%\qquad
%\deduct
%    {\caseVarRcd{\_:\tau_0}{\rho}{\tau}}
%    {\caseVarRcd{\_:\tau_0}{l:\tau_l,\rho}{\tau}}
%    {}
\]
The third rule above is meant to capture the idea that a variant type
with a default label can only be ``eliminated'' by a record type with a
default label.

\subsection{Typing Rules \;($\hastp{\G}{e}{\tau}$)}

\textbf{Quantifiers}
\[
\deduct
    {\hastp{\G}{e}{\tau[\alpha := p]}}
    {\hastp{\G}{e}{\forall \alpha . \tau}}
    {(p\;\text{fresh})}
\qquad
\deduct
    {\hastp{\G}{e}{\forall \alpha.\tau}}
    {\hastp{\G}{e}{\tau[\alpha := \tau']}}
    {}
\]

\noindent
\textbf{Functions and bound variables}
\[
\deduct
    {x:\tau \in \G}
    {\hastp{\G}{x}{\tau}}
    {}
\qquad
\deduct
    {\hastp{\G,x:\tau_x}{e}{\tau}}
    {\hastp{\G}{\lam{x}{e}}{\tau_x\to\tau}}
    {}
\qquad
\deduct
    {\hastp{\G}{e_0}{\tau_1\to\tau}
     \quad
     \hastp{\G}{e_1}{\tau_1}
    }
    {\hastp{\G}{\app{e_0}{e_1}}{\tau}}
    {}
\]

\noindent
\textbf{Records}
\[
\deduct
    {}
    {\hastp{\G}{\{\cdot\}}{\{\cdot\}}}
    {}
\qquad
\deduct
    {\hastp{\G}{e}{\tau}}
    {\hastp{\G}{\{\_:e\}}{\{\_:\tau\}}}
    {}
\]

\[
\deduct
    {\hastp{\G}{e}{\tau}
      \quad
      \hastp{\G}{\{r\}}{\{\rho\}}
    }
    {\hastp{\G}{\{l:e,r\}}{\{l:\tau,\rho\}}}
    {}
\qquad
\deduct
    {\hastp{\G}{e}{\{\rho\}}
     \qquad
     \haslbl{\rho}{l}{\tau}
    }
    {\hastp{\G}{e\#l}{\tau}}
    {}
\]

\noindent
\textbf{Variants}
\[
\deduct
    {\hastp{\G}{e}{\tau}}
    {\hastp{\G}{\variant{\_:e}}{\variant{\rho,\_:\tau}}}
    {}
\qquad
\deduct
    {\hastp{\G}{e}{\variant{\rho_0,l:\tau,\rho_1,\_:\tau}}}
    {\hastp{\G}{e}{\variant{\rho_0,\rho_1,\_:\tau}}}
    {(l\not\in\rho_0,\rho_1)}
\]

\[
\deduct
    {\hastp{\G}{e}{\tau}}
    {\hastp{\G}{\variant{l:e}}{\variant{\rho_0,l:\tau,\rho_1}}}
    {(l\not\in\rho_0)}
\qquad
\deduct
    {\hastp{\G}{e_0}{\variant{\rho_0}}
     \quad
     \hastp{\G}{e_1}{\{\rho_1\}}
     \quad
     \caseVarRcd{\rho_0}{\rho_1}{\tau}
    }
    {\hastp{\G}{\case{e_0}{e_1}}{\tau}}
    {}
\]

\subsection{Examples}
(assume some base types and corresponding constants)

Here are some expressions with their (not necessarily unique) deriveable types:
\[\begin{array}{lcl}
\case
    {\variant{b:5}}
    {\{a:\lam{x}{x + 1},\_:\lam{x}{x * 0}\}}
& : &
\tt{Num}
\\
\lam
    {r}
    {\case
      {\variant{b:\tt{``b"}}}
      {r}
    }
& : &
\{a:{\tt{Num}}\to{\tt{Num}},\_:\forall\alpha.{\alpha}\to{\tt{Num}}\}
\to
\tt{Num}
\\
\variant{b:\tt{``b"}} & : & \variant{a:\{\},\_:\tt{Text}}
\\
\case
  {\variant{b:\tt{``b"}}}
  {\{a:\lam{x}{x + 1},\_:\lam{x}{0}\}}
& : &
\tt{Num}
\\
\case
    {\variant{\_:\tt{``b"}}}
    {\{a:\lam{x}{x + 1},\_:\lam{x}{0}\}}
& : &
\tt{Num}
\end{array}
\]

Here are some untypeable expressions:
\[\begin{array}{l}
 \case
   {\variant{b:\tt{``b"}}}
   {\{a:\lam{x}{x + 1},\_:\lam{x}{x * 0}\}}
\\
 \case
   {\variant{\_:\tt{``b"}}}
   {\{a:\lam{x}{x + 1},b:\lam{x}{0}\}}
\end{array}
\]


\section{Evaluation}

\subsection{Term Label Lookup \;($\haslbltm{r}{l}{e}$)}
\[
\deduct
    {}
    {\haslbltm{\_ : e}{l}{e}}
    {}
\qquad
\deduct
    {}
    {\haslbltm{l:e,r}{l}{e}}
    {}
\qquad
\deduct
    {\haslbltm{r}{l}{e}}
    {\haslbltm{l':e',r}{l}{e}}
    {(l \neq l')}
\]

\subsection{Evaluation Rules  \;($\eval{e_0}{e_1}$)}
\[
\deduct
    {}
    {\eval{\lam{x}{e}}{\lam{x}{e}}}
    {}
\qquad
\deduct
    {\eval{e_0}{\lam{x}{e'}}
     \quad
     \eval{e'[x:=e_1]}{e}
    }
    {\eval{\app{e_0}{e_1}}{e}}
    {}
\]

\[
\deduct
    {}
    {\eval{\{r\}}{\{r\}}}
    {}
\qquad
\deduct
    {\eval{e}{\{r\}}
     \quad
     \haslbltm{r}{l}{e_l}
     \quad
     \eval{e_l}{e'}
    }
    {\eval{e\#l}{e'}}
    {}
\]

\[
\deduct
    {}
    {\eval{\variant{l:e}}{\variant{l:e}}}
    {}
\qquad
\deduct
    {\eval{e_0}{\variant{l:e_l}}
     \quad
     \eval{e_1}{\{r\}}
     \quad
     \haslbltm{r}{l}{e_r}
     \quad
     \eval{\app{e_r}{e_l}}{e}
    }
    {\eval{\case{e_0}{e_1}}{e}}
    {}
\]

\section{Correctness}

\begin{theorem}
$\forall e \,\tau . \exists e' . \;\;\hastp{\cdot}{e}{\tau}$ and $\eval{e}{e'}\;$ implies $\;\hastp{\cdot}{e'}{\tau}$
\end{theorem}

\section{A Tighter Coupling of Records and Variants}
Record terms are currently used to consume variant terms, i.e. in
$\case{e_0}{e_1}$, the second argument $e_1$ should be a record;
however there is a bit of lost symmetry since variant terms are not
involved in consuming records. In this section we slightly reformulate
the system to allow this symmetry. Specifically we will use projection
as the single elimination form for both records and variants, and we
will modify projection to take a variant as a second argument, rather
than a label; this change also removes the reliance on raw labels,
which are not first class values, in the core expressions.
\\

\noindent
\textbf{Variants}
\\
No $\tt{case}$ construct.
\\

\subsection{Project Fun \;($\projFunRule{\rho_0}{\rho_1}{\tau}$)}
\label{subsec:case-constraint}
\[
\deduct
    {\haslbl{\rho_0}{l_1}{\tau_1\to\tau} \quad \projFunRule{\rho_0}{\rho_1}{\tau}}
    {\projFunRule{\rho_0}{l_1:\tau_1,\rho_1}{\tau}}
    {}
\qquad
\deduct
    {}
    {\projFunRule{\rho}{\cdot}{\tau}}
    {}
\qquad
\deduct
    {\haslbl{\rho}{l}{\tau_0\to\tau}}
    {\projFunRule{\rho}{\_:\tau_0}{\tau}}
    {(l\not\in\rho)}
\]
Third rule above is interpretation of term variant default label as
only being able to project the term record default label; removing the
condition would change this to allowing the (non-deterministic)
projection of any suitable label.

\subsection{Project Arg \;($\projArgRule{\rho_0}{\rho_1}{\tau}$)}
\label{subsec:case-constraint}
\[
\deduct
    {\haslbl{\rho_0}{l_1}{\tau_1} \quad \projArgRule{\rho_0}{\rho_1}{\tau}}
    {\projArgRule{\rho_0}{l_1:\tau_1\to\tau,\rho_1}{\tau}}
    {}
\qquad
\deduct
    {}
    {\projArgRule{\rho}{\cdot}{\tau}}
    {}
\qquad
\deduct
    {\haslbl{\rho}{l}{\tau_0\to\tau}}
    {\projArgRule{\rho}{\_:\tau_0}{\tau}}
    {(l\not\in\rho)}
\]
\\

%%%%%%%%%%%%%%%%%%%%%%%%%%%%%%%
\ignore{
\noindent
\textbf{New types}
\[\begin{array}{rcl}
\sigma & \define & \alpha \mid \forall \alpha. \sigma \mid \tau\to\tau
\end{array}
\]

\noindent
\textbf{Records}
\[
\deduct
    {}
    {\hastp{\G}{\{\cdot\}}{\{\cdot\}}}
    {}
\qquad
\deduct
    {\hastp{\G}{e}{\sigma}}
    {\hastp{\G}{\{\_:e\}}{\{\_:\sigma\}}}
    {}
\qquad
\deduct
    {\hastp{\G}{e}{\sigma}
      \quad
      \hastp{\G}{\{r\}}{\{\rho\}}
    }
    {\hastp{\G}{\{l:e,r\}}{\{l:\sigma,\rho\}}}
    {}
\]
}
%%%%%%%%%%%%%%%%%%%%%%%%%%%%%%%

%% \[
%% \deduct
%%     {\hastp{\G}{e}{\tau}}
%%     {\hastp{\G}{\variant{l:e}}{\variant{\rho_0,l:\tau,\rho_1}}}
%%     {(l\not\in\rho_0)}
%% \]

%% \[
%% \deduct
%%     {\hastp{\G}{e}{\tau}}
%%     {\hastp{\G}{\variant{\_:e}}{\variant{\rho,\_:\tau}}}
%%     {}
%% \qquad
%% \deduct
%%     {\hastp{\G}{e}{\variant{\rho_0,l:\tau,\rho_1,\_:\tau}}}
%%     {\hastp{\G}{e}{\variant{\rho_0,\rho_1,\_:\tau}}}
%%     {(l\not\in\rho_0,\rho_1)}
%% \]

\noindent
\textbf{Row Elimination (Projection)}
\[
\deduct
    {\hastp{\G}{e_0}{\{\rho_0\}}
     \quad
     \hastp{\G}{e_1}{\variant{\rho_1}}
     \quad
     \projFunRule{\rho_0}{\rho_1}{\tau}
    }
    {\hastp{\G}{e_0\projFun e_1}{\tau}}
    {}
\qquad
\deduct
    {\hastp{\G}{e_0}{\{\rho_0\}}
     \quad
     \hastp{\G}{e_1}{\variant{\rho_1}}
     \quad
     \projArgRule{\rho_0}{\rho_1}{\tau}
    }
    {\hastp{\G}{e_0\projArg e_1}{\tau}}
    {}
\]

\noindent
\textbf{Projection Evaluation}
\[
\deduct
    {\eval{e_0}{\{r\}}
     \quad
     \eval{e_1}{\variant{l:e_l}}
     \quad
     \haslbltm{r}{l}{e_r}
     \quad
     \eval{\app{e_r}{e_l}}{e}
    }
    {\eval{e_0\projFun e_1}{e}}
    {}
\qquad
\deduct
    {\eval{e_0}{\{r\}}
     \quad
     \eval{e_1}{\variant{l:e_l}}
     \quad
     \haslbltm{r}{l}{e_r}
     \quad
     \eval{\app{e_l}{e_r}}{e}
    }
    {\eval{e_0\projArg e_1}{e}}
    {}
\]

\section{Small-Step Semantics}

These semantics are agnostic about whether we are using default labels. If not, we could simplify
the system by writing the typing and evaluation rules without using special lookup judgments
(e.g., ``$\haslbltm{r}{l}{e_r}$'' could be replaced by ``$r = r_1, l{:}e_r, r_2$ where $l\not\in r_1$''.)

% We use $v$ to refer to a value and $r_v$ to refer to the contents of a record value, e.g.,

% \[\begin{array}{rcl}
% v & \define & \lam{x}{e} \mid \{r_v\} \variant{l:v}
% \\
% r_v & \define & \cdot \mid \_ : v \mid l : v, r
% \\
% \end{array}
% \]

\subsection{Core Rules}
\[
\deduct
    {\step{e_0}{e'_0}}
    {\step{\app{e_0}{e_1}}{\app{e'_0}{e_1}}}
    {}
\qquad
%\deduct
%    {\step{e_1}{e'_1}}
%    {\step{\app{v_0}{e_1}}{\app{v_0}{e'_1}}
%    {}
%\qquad
\deduct
    {}
    {\step{\app{(\lam{x}{e'})}{e_1}}{e'[x:=e_1]}}
    {}
\]

\subsection{If we are using projection}

\[
\deduct
    {\step{e_0}{e'_0}}
    {\step{e_0\#l}{e'_0\#l}}
    {}
\qquad
\deduct
    {\haslbltm{r}{l}{e_l}}
    {\step{\{r\}\#l}{e_l}}
    {}
\]

\[
\deduct
    {\step{e_0}{e'_0}}
    {\step{\case{e_0}{e_1}}{\case{e'_0}{e_1}}}
    {}
\qquad
\deduct
    {\step{e_1}{e'_1}}
    {\step{\case{e_0}{e_1}}{\case{e_0}{e'_1}}}
    {}
\qquad
\deduct
    {\haslbltm{r}{l}{e_r}}
    {\step{\case{\variant{l:e_l}}{\{r\}}}{\app{e_r}{e_l}}}
    {}
\]

We could enforce left-to-right evaluation by having the second ``case'' rule
start working on $e_1$ only if $e_0$
is already a value, but in the absence of side-effects it doesn't matter.

\subsection{If we are using the symmetric version:}

\[
\deduct
   {\step{e_0}{e'_0}}
   {\step{e_0 \projFun e_1}{e'_0 \projFun e_1}}
   {}
\qquad
\deduct
   {\step{e_1}{e'_1}}
   {\step{e_0 \projFun e_1}{e_0 \projFun e'_1}}
   {}
\qquad
\deduct
   {\haslbltm{r}{l}{e_r}}
   {\step{\{r\} \projFun \variant{l:e_l}}{\app{e_r}{e_l}}}
   {}
\]

\[
\deduct
   {\step{e_0}{e'_0}}
   {\step{e_0 \projArg e_1}{e'_0 \projArg e_1}}
   {}
\qquad
\deduct
   {\step{e_1}{e'_1}}
   {\step{e_0 \projArg e_1}{e_0 \projArg e'_1}}
   {}
\qquad
\deduct
   {\haslbltm{r}{l}{e_r}}
   {\step{\{r\} \projArg \variant{l:e_l}}{\app{e_l}{e_r}}}
   {}
\]

\section{Small-Step Correctness}

Note: the following lemmas might not hold (or the proofs might be harder) in the presence of
implicit $\forall$ generalization/instantiation. I recommend doing the proofs without
the $\forall$-related typing rules, at least to start.

\todo {check that proofs work with defaults} (will mostly require extending
lemmas and some logic involving case constraints).

\bigskip


We use $v$ to refer to a value. We assume call-by-name, so that record values and
variant values can contain "unevaluated" expressions.

\[\begin{array}{rcl}
v & \define & \lam{x}{e} \mid \{r\} \mid \variant{l:e}\\
\end{array}
\]


\begin{lemma}{Canonical Forms} \label{lem:canonical-forms}
\mbox{}
\begin{enumerate}
\item If $\hastp{\cdot}{v}{\tau_1\to\tau_2}$ then $v$ is $\lam{x}{e}$.
\item If $\hastp{\cdot}{v}{\{\rho\}}$ then $v$ is $\{r\}$.
\item If $\hastp{\cdot}{v}{\variant{\rho}}$ then $v$ is $\variant{l:e}$.
\end{enumerate}
\end{lemma}

\begin{proof}
Without polymorphism, it follows directly by inspection of the typing rules. (E.g.,
the only typing rule that lets us conclude a value has an arrow type is the rule for
typing lambdas, so any value with an arrow type must be a lambda.)
\end{proof}

\begin{theorem}[Progress]
if $\hastp{\cdot}{e}{\tau}$ then either $e$ is a value or else $\step{e}{e'}$ for some $e'$.
\end{theorem}

\begin{proof}
By (structural) induction on the proof that $\hastp{\cdot}{e}{\tau}$.

If \(e\) is a value, then the proof is complete.

If \(e\) is not a value, we examine the proof tree for \(\hastp \cdot e \tau\).

We consider the possible cases for \(e\), ignoring ones that are values.

\begin{itemize}
\item Case \(\tau = \forall \alpha. \tau'\). By lemma \ref{lem:inversion}, we
  know that \(\hastp \cdot e {\tau'[\alpha := p]}\) for fresh \(p\). We may then
  continue by proving the statement for this type (removing as many prenex
  quantifiers in this fashion until we hit one of the below cases).
\item Case \(e_0 e_1\).
    By lemma \ref{lem:inversion}, we know that \(\hastp \cdot {e_0} {\tau_1 \to
    \tau}\) and \(\hastp \cdot {e_1} \tau_1\).

    There are two cases to consider for \(e_0\).

    If \(e_0\) is a value, then by lemma \ref{lem:canonical-forms} we conclude
    that \(e_0 = \lambda x. e_0'\) for some \(e_0'\). We may then apply the
    \(\step {(\lambda x.e_0') e_1} {e_0'[x:=e_1]}\) rule. So for this case, we
    conclude \(\step e {e'}\) for an \(e' = e_0'[x:=e_1]\).

    If \(e_0\) is not a value, then by the induction hypothesis we know that
    there exists some \(e_0'\) such that \(\step e_0 {e_0'}\). We may then apply
    the \(\step {e_0 e_1}{e_0' e_1}\) rule. For this case, we conclude \(\step e
    {e'}\) for an \(e' = e_0' e_1\).

  \item Case \(e_0 \# \ell\). By lemma \ref{lem:inversion}, we know that
    \(\hastp \cdot {e_0} {\{\rho\}}\) and that \(\haslbl \rho \ell \tau\).

    If \(e_0\) is a value, then by lemma \ref{lem:canonical-forms}, \(e_0\) is
    some record \(\{r\}\). By lemma \ref{lem:row-type-term}, \(\haslbltm r \ell
    {e_0'}\). We may then apply the \(\step {\{r\} \# \ell} {e_0'}\) rule. For this
    case, we conclude \(\step e {e_0'}\).

    If \(e_0\) is not a value, then by the induction hypothesis we know that
    there exists some \(e_0'\) such that \(\step e_0 {e_0'}\). We may then apply
    the \(\step {e_0 \# \ell}{e_0' \# \ell}\) rule. For this case, we conclude
    \(\step e {e'}\) for an \(e' = e_0' \# \ell\).

  \item Case \(\case e_0 e_1\). By lemma \ref{lem:inversion}, we know \(\hastp
    \cdot {e_0} {\variant {\rho_0}}\), \(\hastp \cdot {e_1} {\{\rho_1\}}\),
    and \(\caseVarRcd {\rho_0} {\rho_1} \tau\).

    If either \(e_0\) or \(e_1\) is not a value, we may proceed as in the
    non-value cases of the application or projection parts of the proof by using
    the induction hypothesis and the corresponding step rule.

    If both \(e_0\) and \(e_1\) are values, we conclude from lemma
    \ref{lem:canonical-forms} that \(e_0 = \variant{\ell : e_0'}\) and \(e_1 =
    \{r\}\). Examining the rules for case constraint, it must be the case that
    \(\haslbl {\rho_1} \ell {\tau_0 \to \tau}\) because \(\rho_0\) is nonempty
    (this is trivially true because \(e_0\) is a value with type \(\variant
    {\rho_0}\)). The only way for \(\rho_1\) to not have label \(\ell\) would be
    for \(\rho_0\) to not have label \(\ell\). By lemma \ref{lem:row-type-term},
    it must be the case that \(\haslbltm{r}{\ell}{e_1'}\). Therefore, we may
    apply rule \(\step {\case {\variant {\ell : e_0'}} {\{r\}}} {e_1' e_0'}\)
    and this case is complete.

\end{itemize}

These cases cover all of the non-values. Since the values constitute the base
cases of the structural induction, and the inductive hypothesis holds, the proof
is complete.

\end{proof}

\begin{lemma}{Inversion} \label{lem:inversion}
Note that all cases aside from 1 apply only if \(\tau\) is not quantified.
\mbox{}
\begin{enumerate}
\item If \(\hastp \Gamma e {\forall \alpha. \tau},\) then for fresh \(p\), \(\hastp \Gamma e
  {\tau[\alpha := p]}\).
\item If $\hastp{\Gamma}{\app{e_0}{e_1}}{\tau}$ then
  $\hastp{\Gamma}{e_0}{\tau_1\to\tau}$ and $\hastp{\Gamma}{e_1}{\tau_1}$ for some type $\tau_1$.
\item If $\hastp{\Gamma}{\lam{x}{e}}{\tau}$ then $\tau = (\tau_1\to\tau_2)$ for
some types $\tau_1$ and $\tau_2$, and
  $\hastp{\Gamma,x:\tau_1}{e}{\tau_2}$.
\item If \(\hastp \Gamma {e \# \ell} \tau\), then \(\hastp \Gamma e {\{\rho\}}\)
  and \(\haslbl \rho \ell \tau\).
\item If \(\hastp \Gamma {\case e_0 e_1} \tau\), then \(\hastp \Gamma {e_0}
  {\variant {\rho_0}}\), \(\hastp \Gamma {e_1} {\{\rho_1\}}\), and \(\caseVarRcd
  {\rho_0} {\rho_1} \tau\).
\item If \(\hastp \Gamma {\variant {\ell : e}} {\variant \rho}\), then \(\haslbl
  \rho \ell t\) and \(\hastp \Gamma e \tau\).
\item If \(\hastp \Gamma {\lambda x. e} {\tau_0 \to \tau}\), then \(\hastp
  {\Gamma, x : \tau_0} e \tau\).
\item Other facts as needed\ldots
\end{enumerate}
\end{lemma}

\begin{proof}
By inspection of the inference rules. (I.e., if we have a complete typing proof,
and there is only one possible typing rule that can give us the corresponding conclusion,
then the proofs of the premises of that rule must be contained within our assumed typing proof.)

The variant case can either come from a deault variant derivation or a regular
variant derivation. The latter follows immediately from the rules. The former
follows from the fact that if a label is in the variant term but not in its
type, its expression must have the default type.

\todo{this can be fleshed out/formalized a bit}

\end{proof}

\begin{lemma}{Row Type-Term Correspondance} \label{lem:row-type-term}
  \mbox{}
  \begin{enumerate}
  \item If \(\hastp \Gamma {\{r\}} {\{\rho\}}\), then \(\haslbl \rho \ell \tau\)
    if and only if \(\haslbltm r \ell e\) and \(\hastp \Gamma e \tau\).
  \item If \(\hastp \Gamma {\variant {\ell : e}} {\variant \rho}\), then
    \(\haslbl \rho \ell \tau\) and \(\hastp \Gamma e \tau\).
  \end{enumerate}
\end{lemma}

\begin{proof}
  The proofs follow from the respective rules.
  \mbox{}
  \begin{enumerate}
  \item \((\Rightarrow)\) Suppose that \(\haslbl \rho \ell \tau\). There are two
    cases. Note that we ignore the quantifier rule since if we conclude that
    \(\hastp \Gamma e {\forall \alpha. \tau}\) when we really wanted \(\hastp
    \Gamma e {\tau[\alpha := \tau']}\), we may simply use the forall elimination
    derivation to get this result.
    \begin{itemize}
    \item \(\ell\) is in \(\rho\). Then for \(\{r\}\) to have type \(\{\rho\}\), it
      must contain \(\ell\) and furthermore \(e\) must have type \(\tau\). This
      follows from the fact that only one rule could result in \(\{r\}\) having
      type \(\{\rho\}\).
    \item \(\ell\) is not in \(\rho\). Then for \(\{r\}\) to have type
      \(\{\rho\}\), it must contain a default and furthermore the \(e\)
      corresponding to this default must have type \(\tau\). This again follows
      from the fact that there is one rule that could result in \(\{r\}\) having
      type \(\{\rho\}\).
    \end{itemize}

    \((\Leftarrow)\) This follows in a similar manner as the forward direction,
    except with the logic reversed (observe that the term and type level lookups
    are almost identical; they just deal with different objects).
  \item This follows from lemma \ref{lem:inversion}.
  \end{enumerate}
\end{proof}

\begin{theorem}[Preservation]
If $\hastp{\cdot}{e}{\tau}$ and $\step{e}{e'}$ then $\hastp{\cdot}{e'}{\tau}$.
\end{theorem}


\begin{proof}
  By (structural) induction on the term $e$ and typing derivation \(\hastp \cdot
  e \tau\). We consider all of the possible forms that \(e\) can take.

\textit{Dealing with quantifiers.} This section is probably no longer necessary.

For each of these rules, whenever we encounter a prenex quantifier in the
\(\tau\), we cannot use lemma \ref{lem:inversion} to directly get the types we
want. Instead, we use it on the quantifier(s) until we reach the first \(\tau'\)
with no prenex quantifiers. This means that \(\tau'\) will have the same inner
structure as \(\tau\), only with some fresh free variables \(p_1, p_2, \dots\)
in place of the quantified variables \(\alpha_1, \alpha_2, \dots\). This is
because the only way to introduce a quantifier is to have a typing derivation
with fresh free variables (see the quantifier introduction rule). When we invoke
the induction hypothesis, the new term \(e'\) has type \(\tau'\). This is fine,
since we know that under the empty context, \(\tau\) may be derived from
\(\tau'\), so we may do just that.

Symbolically, say we have some \(\hastp \cdot e {\forall \alpha.
  \tau}\). By lemma \ref{lem:inversion}, the derivation for \(\forall \alpha.
\tau\) looks like

\[
\deduct
    {\hastp{\cdot}{e}{\tau[\alpha := p]}}
    {\hastp{\cdot}{e}{\forall \alpha . \tau}}
    {(p\;\text{fresh})}
\]

Let \(\tau' = \tau[\alpha := p]\). Suppose \(\tau'\) has no prenex quantifier
(otherwise, we'd keep using lemma \ref{lem:inversion}). Now we may apply one of
the below rules on \(\hastp \cdot e {\tau'}\) depending on what form \(\step e
{e'}\) takes. The application of one of these rules lets us conclude \(\hastp
\cdot {e'} {\tau'}\). At this point we're done, since we know that we can
perform the inverse of the derivation we used to get \(\tau'\) from \(\tau\) and
reintroduce the quantifier.

\begin{itemize}
  \item Case \(\tau = \forall \alpha. \tau'\). If there is a leading quantifier,
    we may apply lemma \ref{lem:inversion} to conclude that \(\hastp \cdot e
    {\tau'[\alpha := p]}\) for some fresh \(p\). We may then apply the induction
    hypothesis to conclude that \(\hastp \cdot {e'} {\tau'[\alpha := p]}\). From
    this point, it only takes one step (quantifier introduction) to conclude
    \(\hastp \cdot {e'} {\forall \alpha. \tau'}\) as desired.
  \item Case \(\step {e_0 e_1} {e_0' e_1}\). Since \(\hastp \cdot {e_0 e_1}
    \tau\), lemma \ref{lem:inversion} means that \(\hastp \cdot {e_0} {\tau_0
      \to \tau}\) and \(\hastp \cdot {e_1} {\tau_0}\).

    By the induction hypothesis, \(\hastp \cdot {e_0'} {\tau_0 \to \tau}\). This
    allows us to use the application typing rule to derive that \(\hastp \cdot
    {e_0' e_1} \tau\), as desired.
  \item Case \(\step {(\lambda x. e')e_1} {e'[x:=e_1]}\). Like the above, since
    \(\hastp \cdot {(\lambda x.e') e_1} \tau\), lemma \ref{lem:inversion} means
    that \(\hastp \cdot {(\lambda x.e')} {\tau_0 \to \tau}\) and \(\hastp \cdot
    {e_1} {\tau_0}\). We may apply lemma \ref{lem:inversion} again to conclude
    that \(\hastp {x : \tau_0} {e'} \tau\). Lemma \ref{lem:substitution} gives
    that \(\hastp \cdot {e'[x:=e_1]} \tau\), as desired.
  \item All other cases involving steps in subparts of the equation have proofs
    following the same direction (appeal to lemma \ref{lem:inversion} followed
    by application of the induction hypothesis, then an amendment to the
    derivation of \(\hastp \cdot e \tau\) that results in \(\hastp \cdot {e'}
    \tau\)).
  \item Case \(\step {\{r\} \# \ell} {e_\ell}\). By lemma \ref{lem:inversion},
    we conclude that \(\hastp \cdot e {\{\rho\}}\) and \(\haslbl \rho \ell
    \tau\). From this, we may use lemma \ref{lem:row-type-term} to conclude that
    \(\hastp \cdot {e_\ell} \tau\), as desired.
  \item Case \(\step {\case {\variant {\ell : e_\ell}} {\{r\}}} {e_r e_\ell}\).
    By lemma \ref{lem:inversion}, we conclude that \(\hastp \cdot {\variant
      {\ell : e_\ell}} {\variant {\rho_0}}\), \(\hastp \cdot {\{r\}}
    {\{\rho_1\}}\), and \(\caseVarRcd {\rho_0} {\rho_1} \tau\). By lemma
    \ref{lem:row-type-term}, we know that \(\hastp \cdot {e_\ell} {\tau_\ell}\)
    and \(\haslbl {\rho_0} \ell {\tau_\ell}\). Because \(\caseVarRcd {\rho_0}
    {\rho_1} \tau\), it must be that \(\haslbl {\rho_1} \ell {\tau_\ell \to
      \tau}\). By lemma \ref{lem:row-type-term}, we know that \(\haslbltm r \ell
    {e_r}\) and \(\hastp \cdot {e_r} {\tau_\ell \to \tau}\). The types of
    \(e_\ell\) and \(e_r\) allow us to derive \(\hastp \cdot {e_\ell e_r}
    \tau\), as desired.
\end{itemize}

\end{proof}

\begin{lemma}[Substitution] \label{lem:substitution} If \(\hastp {\Gamma_0, x :
    \tau_x, \Gamma_1} e \tau\) and \(\hastp {\Gamma_0, \Gamma_1} {e'}
  {\tau_x}\), then \(\hastp {\Gamma_0, \Gamma_1} {e[x:=e']} \tau\).
\end{lemma}

\begin{proof}
  % Follows by induction on the structure of \(e\). I think. \todo{write proof}

  % For the base case (variable case), can use inversion to conclude that if the
  % variable is not equal to the one subsituted, it must be in either \(\Gamma_0\)
  % or \(\Gamma_1\) and use that to conclude it works (avoids a weakening lemma).

  We will modify the typing derivation \(\hastp {\Gamma_0, x : \tau_x, \Gamma_1}
  e \tau\) so that it becomes \(\hastp {\Gamma_0, \Gamma_1} {e[x:=e']} \tau\).
  This modification can happen recursively, and in most cases all it requires is
  recursive traversals or empty base cases. The only missing case is that for
  variables.

  \[
    \deduct
      {x : {\tau_x} \in \Gamma}
      {\hastp \Gamma x {\tau_x}}
  \]

  Since the substitution will always replace \(x\) with \(e'\), we may modify
  this case to simply be the proof \(\hastp \Gamma {e'} {\tau_x}\), which is a
  premise. Since this is the only place where \(x : \tau_x\) is used in the
  context, it's safe to be removed from the updated context.

  \[
    \deduct
      {y : \tau \in \Gamma}
      {\hastp \Gamma y \tau}
      {(y \ne x)}
  \]

  In this case, nothing needs to happen since no substitution will happen in
  \(y\). And since \(y : \tau\) is not \(x : \tau_x\), it will not be removed
  from the context of the resulting derivation.
\end{proof}

\begin{theorem}[Soundness]
If $\hastp{\cdot}{e}{\tau}$ then either $\steps{e}{v}$ for some result value $v$
(with $\hastp{\cdot}{v}{\tau}$), or
there is an infinite sequence $e \leadsto e_1 \leadsto e_2 \leadsto \cdots$.
\end{theorem}

\begin{proof}
Direct consequence of Preservation and Progress.
\end{proof}


\end{document}
